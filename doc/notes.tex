\documentclass[onecolumn, prb,preprintnumbers,amsmath,amssymb,floatfix]{revtex4}
\usepackage[linktocpage,bookmarksopen,bookmarksnumbered]{hyperref}
%\usepackage[backend=biber,style=nature]{biblatex}

\usepackage{graphicx}
\usepackage{dcolumn}

\bibliographystyle{naturemag}

\usepackage{amsmath,graphics,epsfig,color,verbatim,ulem}
\usepackage{braket}
\newcommand{\eps}{\epsilon} \renewcommand{\a}{\alpha}
\renewcommand{\b}{\beta} \newcommand{\vR}{{\mathbf{R}}}
\renewcommand{\vr}{{\mathbf{r}}} 
\newcommand{\vk}{{\mathbf{k}}}
\newcommand{\vb}{{\mathbf{b}}}
\newcommand{\vp}{{\mathbf{p}}}
\newcommand{\vG}{{\mathbf{G}}}
\newcommand{\vK}{{\mathbf{K}}} \newcommand{\vq}{{\mathbf{q}}}
\newcommand{\vQ}{{\mathbf{Q}}} \newcommand{\vPhi}{{\mathbf{\Phi}}}
\newcommand{\vS}{{\mathbf{S}}} \newcommand{\cG}{{\cal G}}
\newcommand{\cF}{{\cal F}} \newcommand{\cD}{{\cal D}}
\newcommand{\Tr}{\mathrm{Tr}} \newcommand{\npsi}{\underline{\psi}}
\newcommand{\vA}{{\mathbf{A}}} \newcommand{\vE}{{\mathbf{E}}}
\newcommand{\vj}{{\mathbf{j}}} \newcommand{\vv}{{\mathbf{v}}}
\newcommand{\kb}{k_B} \newcommand{\cellvol}{}
\newcommand{\trace}{\mbox{Tr}} \newcommand{\ra}{\rangle }
\newcommand{\la}{\langle } \newcommand{\om}{\omega}
\renewcommand{\Im}{\mathrm{Im}} \newcommand{\up}{\uparrow}
\newcommand{\down}{\downarrow}
\newcommand{\nphi}{\underline{\phi}}
\newcommand{\tIm}{\overline{\Im}}
\newcommand{\cb}{{\cal B}}
\usepackage{multirow}

\addtolength{\itemsep}{-0.05in}

\usepackage{tabularx,ragged2e,booktabs,caption}
\usepackage{cancel}
\newcolumntype{C}[1]{>{\Centering}m{#1}}
\renewcommand\tabularxcolumn[1]{C{#1}}
\usepackage{multibib}
%\addbibresource{biblatex-nature.bib}

%% the natbib package works better than cite
%\usepackage[square,numbers,comma,sort&compress]{natbib}
%\usepackage[square,numbers,sort]{natbib}

\begin{document}
\special{papersize=8.5in,11in}
\setlength{\pdfpageheight}{\paperheight}
\setlength{\pdfpagewidth}{\paperwidth}
% You may need to change the horizontal offset to do what you
% want.  Setting \hoffset to a negative value moves all printed
% material to the left on all pages; setting it to a positive value
% moves all printed material to the right on all pages; not setting
% it keeps all printed material in it's default position.  \voffset
% is the vertical offset: use negative value for up; don't set if
% you want to use default position; use positive for down.
% \hoffset = -0.2truein
% \voffset = -0.2truein

\title{Quantum Geometric Tensor notes}

Quantum Geometric Tensor is defined by
\begin{eqnarray}
&&  M_{\mu\nu}^{\cb}(\vk)  =2\sum_{i\in \cb}
\braket{\frac{\partial}{\partial k_{\mu}}\psi_{\vk i}|(1-\sum_{j\in \cb}\ket{\psi_{\vk j}}\bra{\psi_{\vk j}})|\frac{\partial}{\partial k_{\nu}}\psi_{\vk i}}\\
&&  g_{\mu\nu}(\vk) = \frac{1}{2}(M_{\mu\nu}(\vk)+M_{\nu\mu}(\vk))\\
\end{eqnarray}
The Berry curvature is
\begin{equation}
\Omega_{\mu\nu} = i(M_{\mu\nu}-M_{\nu\mu}) 
\end{equation}

How is quantum metric related to distance between Bloch states? We have
\begin{eqnarray}
1-|\braket{\psi_{\vk n}|\psi_{\vk+d\vk,n}}|^2 =\frac{1}{2}\sum_{\mu\nu} M_{\mu\nu}^n(\vk) dk_\mu dk_\nu
\end{eqnarray}  
To derive this, you need to take into account both the first and the second derivative of the wave function, i.e.,
\begin{eqnarray}
  \psi_{\vk+d\vk,n}\approx \psi_{\vk,n} +
 \sum_{\mu}\frac{\partial}{\partial k_\mu}\psi_{\vk,n}dk_\mu +
 \frac{1}{2}\sum_{\mu\nu}\frac{\partial^2}{\partial k_\mu\partial k_\nu}\psi_{\vk,n}dk_\mu dk\nu
\end{eqnarray}
and then the straightforward expansion leads to the above formula.


The definition is gauge invariant, in the sense that redefining single particle wave functions by an arbitrary phase
$$\widetilde{\psi}_{\vk i}(\vr)=e^{i\beta_i(\vk)}\psi_{\vk i}(\vr)$$ does not change the result. This is crucial for implementation, as the phase of eigenvectors is arbitrary, chosen by the diagonalization routine.

We check the gauge invariance by
\begin{eqnarray}
  \frac{\partial}{\partial k_{\nu}}\widetilde{\psi}_{\vk i}(\vr)=
  e^{i\beta_i(\vk)} ({i\beta_i(\vk)}+\frac{\partial}{\partial k_{\nu}})\psi_{\vk i}(\vr)
\end{eqnarray}  
which means
\begin{eqnarray}
 \frac{1}{2} M_{\mu\nu}^{\cb}(\vk)  =
\sum_{i,j\in \cb}
  \braket{\frac{\partial}{\partial k_{\mu}}\widetilde{\psi}_{\vk i}|(1-\sum_{j\in \cb}\ket{\widetilde{\psi}_{\vk j}}\bra{\widetilde{\psi}_{\vk j}})|\frac{\partial}{\partial k_{\nu}}\widetilde{\psi}_{\vk i}}=
\\
\braket{({i\beta_i(\vk)}+\frac{\partial}{\partial k_{\mu}}) \psi_{\vk i}|(1-\sum_{j\in \cb}  \ket{\psi_{\vk j}}\bra{\psi_{\vk j}})|  ({i\beta_i(\vk)}+\frac{\partial}{\partial k_{\nu}})\psi_{\vk i}}=\\
=\braket{\frac{\partial}{\partial k_{\mu}} \psi_{\vk i}|(1-\sum_{j\in \cb}  \ket{\psi_{\vk j}}\bra{\psi_{\vk j}})|\frac{\partial}{\partial k_{\nu}}\psi_{\vk i}}\\
+({-i\beta_i(\vk)})({i\beta_i(\vk)})  \braket{\psi_{\vk i}|(1-\sum_{j\in \cb}  \ket{\psi_{\vk j}}\bra{\psi_{\vk j}})| \psi_{\vk i}}\\
{-i\beta_i(\vk)}\braket{ \psi_{\vk i}|(1-\sum_{j\in \cb}  \ket{\psi_{\vk j}}\bra{\psi_{\vk j}})| \frac{\partial}{\partial k_{\nu}}\psi_{\vk i}}\\
+{i\beta_i(\vk)}\braket{\frac{\partial}{\partial k_{\mu}} \psi_{\vk i}|(1-\sum_{j\in \cb}  \ket{\psi_{\vk j}}\bra{\psi_{\vk j}})|\psi_{\vk i}}
\end{eqnarray}
The last three terms vanish because
$$\braket{\phi|(1-\sum_{j\in \cb}  \ket{\psi_{\vk j}}\bra{\psi_{\vk j}})|\psi_{\vk i}}=
\braket{\phi|\psi_{\vk i}}
-\braket{\phi|\psi_{\vk i}}\braket{\psi_{\vk i}|\psi_{\vk i}}=0$$
%
This shows that we could use periodic part of the Bloch functions $u_{\vk i}(\vr)$ instead of $\psi_{\vk i}(\vr)=e^{i\vk\vr}u_{\vk i}(\vr)$. This is crucial when we take finite differences, because
$\braket{\psi_{\vk i}|\psi_{\vk+\vq,i}}=0$ unless $\vq=0$, while 
$\braket{u_{\vk i}|u_{\vk+\vq,i}}\ne 0$.

Next we choose a finite difference approximation for the derivatives, and produce the formula


\begin{eqnarray}
\frac{1}{2} M_{\mu\nu}^{\cb}(\vk)  &=&\frac{1}{\Delta_\mu\Delta_\nu}\sum_{i\in \cb}\braket{ u_{\vk+\Delta_\mu,i}-u_{\vk,i}|(1-\sum_{j\in \cb}\ket{u_{\vk j}}\bra{u_{\vk j}})|u_{\vk+\Delta_\nu i}-u_{\vk,i}}\\
&=&  \frac{1}{\Delta_\mu\Delta_\nu}\sum_{i\in \cb}\braket{ u_{\vk+\Delta_\mu,i}|(1-\sum_{j\in \cb}\ket{u_{\vk j}}\bra{u_{\vk j}})|u_{\vk+\Delta_\nu i}}
\end{eqnarray}

Note that the rest of the terms generated by the above formula all vanish for the same reason as we shown above. The terms are:
\begin{eqnarray}
&-&  \frac{1}{\Delta_\mu\Delta_\nu}\sum_{i\in \cb}\braket{ u_{\vk+\Delta_\mu,i}|(1-\sum_{j\in \cb}\ket{u_{\vk j}}\bra{u_{\vk j}})|u_{\vk,i}}=0\\
&-&  \frac{1}{\Delta_\mu\Delta_\nu}\sum_{i\in \cb}\braket{ u_{\vk,i}|(1-\sum_{j\in \cb}\ket{u_{\vk j}}\bra{u_{\vk j}})|u_{\vk+\Delta_\nu i}}=0\\
&+&  \frac{1}{\Delta_\mu\Delta_\nu}\sum_{i\in \cb}\braket{u_{\vk,i}|(1-\sum_{j\in \cb}\ket{u_{\vk j}}\bra{u_{\vk j}})|u_{\vk,i}}=0
\end{eqnarray}

In cartesian coordinates $\Delta_1=\Delta \vec{e}_{x}$, the above formula takes the form
\begin{eqnarray}
\frac{1}{2} M_{\mu\mu}^{\cb}(\vk)  =
  \frac{1}{\Delta_\mu\Delta_\mu}\sum_{i \in\cb}
  \left(1  -\sum_{j\in\cb} |\braket{ u_{\vk+\Delta_\mu,i}|u_{\vk j}}|^2\right)
\end{eqnarray}
and hence the diagonal components of the quantum geometric tensor are:

\begin{eqnarray}
g_{\mu\mu}(\vk)  =
  \frac{2}{\Delta_\mu\Delta_\nu}\sum_{i \in \cb}
  \left(1  -\sum_{j\in\cb} |\braket{ u_{\vk+\Delta_\mu,i}|u_{\vk j}}|^2\right)
\label{Eq:20}
\end{eqnarray}
The off diagonal terms are
\begin{eqnarray}
g_{\mu\nu}(\vk)  =
  \frac{1}{\Delta_\mu\Delta_\nu}\sum_{i \in\cb}
  \left(
    \braket{u_{\vk+\Delta_\mu,i}|u_{\vk+\Delta_\nu,i}}
  +\braket{u_{\vk+\Delta_\nu,i}|u_{\vk+\Delta_\mu,i}}
  -\sum_{j\in\cb}
  (\braket{ u_{\vk+\Delta_\mu,i}|u_{\vk j}}\braket{u_{\vk j}| u_{\vk+\Delta_\nu,i}}+
  \braket{ u_{\vk+\Delta_\nu,i}|u_{\vk j}}\braket{u_{\vk j}| u_{\vk+\Delta_\mu,i}})
  \right)
\end{eqnarray}
or
\begin{eqnarray}
g_{\mu\nu}(\vk)  =
  \frac{2}{\Delta_\mu\Delta_\nu}\sum_{i \in\cb}
  \textrm{Re}\left(
    \braket{u_{\vk+\Delta_\mu,i}|u_{\vk+\Delta_\nu,i}}
  -\sum_{j\in\cb}
  \braket{ u_{\vk+\Delta_\mu,i}|u_{\vk j}}\braket{u_{\vk j}| u_{\vk+\Delta_\nu,i}}
  \right)
\end{eqnarray}

For degeneracies at momentum point $\vk$ and some set of bands, an arbitrary unitary transformation between these bands should not change the result.
$$\ket{\widetilde{u}_{\vk i}} = \sum_{i'\in dg}U_{i i'}(\vk) \ket{u_{\vk i'}} $$

To make the formula for $g_{\mu\nu}$ invariant, we need to enlarge the space of $\cb$, so that it contains all degenerate bands.
For the diagonal components, it is easy to show that the formula is invariant
\begin{eqnarray}
g_{\mu\mu}(\vk)  =
  \frac{2}{\Delta_\mu\Delta_\mu}\sum_{i' \in \cb}
  \left(1  -\sum_{jj'j''i'i''\in\cb}
%  (U^\dagger(\vk) U(\vk))_{j'' j'} (U(\vk+\Delta_{\mu}) U^\dagger(\vk+\Delta_{\mu}))_{i' i''}
  U_{jj'}(\vk)U^*_{jj''}(\vk) U_{i i'}^*(\vk+\Delta_{\mu})U_{i i''}(\vk+\Delta_\mu)
  \braket{ u_{\vk+\Delta_\mu,i'}|u_{\vk j'}}\braket{u_{\vk j''}| u_{\vk+\Delta_\mu,i''}}\right)
\end{eqnarray}
%
For off-diagonals we have
\begin{eqnarray}
g_{\mu\nu}(\vk)  =
  \frac{2}{\Delta_\mu\Delta_\nu}\sum_{i,i',i''\in \cb}\textrm{Re}
  \left(U^*_{i i'}(\vk+\Delta_\mu)U_{i i''}(\vk+\Delta_\nu)
  \left(\braket{u_{\vk+\Delta_\mu,i'}|u_{\vk+\Delta_\nu,i''}}
  \right.\right.
\\  
  \left.\left.
  -\sum_{j,j',j''\in\cb}
  U^*_{j,j''}(\vk)U_{jj'}(\vk)\braket{ u_{\vk+\Delta_\mu,i'}|u_{\vk j'}}\braket{u_{\vk j''}| u_{\vk+\Delta_\nu,i''}}
  \right)\right)
\end{eqnarray}
The diagonal is clearly invariant, because $U^\dagger U=1$. The off
diagonal terms are not, because we have
$$(U^\dagger(\vk+\Delta_\mu)U(\vk+\Delta_\nu))_{i'i''}$$ and unitary
transformation at different points are not related. This issue arrises
because of our finite difference formula, showing that not every
implementation is guaranteed to work numerically.


The solution is to use symmetrized formula, which is gauge invariant.
We start by writing the symmetrized formula using projectors:
\begin{eqnarray}
g_{\mu\nu}(\vk)  =  \sum_{i j \in\cb}\Tr\left(\partial_\mu P_i(\vk) \partial_\nu P_j(\vk) \right)
\label{Eq:26}
\end{eqnarray}  
where projector is
$$P_i(\vk) =\ket{\psi_{\vk i}}\bra{\psi_{\vk i}}$$
We will first show that this formula is equivalent to above given
formula in the limit where finite difference is turned into a
derivative. After that, we will discretize this formula, and show that
it is invariant to any unitary transformation of gauge.
We start by rewriting Eq.~\ref{Eq:26} into something more commonly
used for quantum metric tensor:
\begin{eqnarray}
&&  g_{\mu\nu}(\vk)  =  \sum_{i j \in\cb}\Tr\left(\partial_\mu P_i(\vk) \partial_\nu P_j(\vk) \right)
\nonumber\\
&&   \sum_{i j \in\cb}\Tr\left(
  \left(\ket{\partial_\mu \psi_{\vk i}}\bra{\psi_{\vk i}}+\ket{\psi_{\vk i}}\bra{\partial_\mu \psi_{\vk i}}\right)
  \left(\ket{\partial_\nu \psi_{\vk j}}\bra{\psi_{\vk j}} +\ket{\psi_{\vk j}}\bra{\partial_\nu\psi_{\vk j}}\right)
   \right)
 \label{Eq:projectorProof}\\
 && =
\sum_{ij\in\cb}\Tr\left(
  \ket{\partial_\mu \psi_{\vk i}}\braket{\psi_{\vk i} |\partial_\nu \psi_{\vk j}}\bra{\psi_{\vk j}} 
  +\ket{\partial_\mu \psi_{\vk i}}\braket{\psi_{\vk i}|\psi_{\vk j}}\bra{\partial_\nu\psi_{\vk j}}
  +\ket{\psi_{\vk i}}\braket{\partial_\mu \psi_{\vk i}|\partial_\nu \psi_{\vk j}}\bra{\psi_{\vk j}}
  +\ket{\psi_{\vk i}}\braket{\partial_\mu \psi_{\vk i}|\psi_{\vk j}}\bra{\partial_\nu\psi_{\vk j}}
  \right)
    \nonumber\\
  &&=
\sum_{ij\in\cb}\Tr\left(
  \braket{\psi_{\vk i} |\partial_\nu \psi_{\vk j}}\braket{\psi_{\vk j}|\partial_\mu \psi_{\vk i}}
  +\braket{\psi_{\vk i}|\psi_{\vk j}}\braket{\partial_\nu\psi_{\vk j}|\partial_\mu \psi_{\vk i}}
  +\braket{\partial_\mu \psi_{\vk i}|\partial_\nu \psi_{\vk j}}\braket{\psi_{\vk j}|\psi_{\vk i}}
  +\braket{\partial_\mu \psi_{\vk i}|\psi_{\vk j}}\braket{\partial_\nu\psi_{\vk j}|\psi_{\vk i}}
     \right)\nonumber\\
  &&=\sum_{ij\in\cb}\left(
    \delta_{ij} (\braket{\partial_\nu\psi_{\vk j}|\partial_\mu \psi_{\vk i}}+\braket{\partial_\mu \psi_{\vk i}|\partial_\nu \psi_{\vk j}})
  + \braket{\psi_{\vk i} |\partial_\nu \psi_{\vk j}}\braket{\psi_{\vk j}|\partial_\mu \psi_{\vk i}}
  + \braket{\partial_\mu \psi_{\vk i}|\psi_{\vk j}}\braket{\partial_\nu\psi_{\vk j}|\psi_{\vk i}}
     \right)\nonumber
 \end{eqnarray}
Because $\partial_\mu (\braket{\psi_{\vk i}|\psi_{\vk j}})=0$ we have
$\braket{\partial_\mu\psi_{\vk i}|\psi_{\vk j}}=-\braket{\psi_{\vk i}|\partial_\mu\psi_{\vk j}}$
hence the last line is
\begin{eqnarray}
  &&  g_{\mu\nu}(\vk)  =\sum_{i\in\cb}   \braket{\partial_\nu\psi_{\vk i}|\partial_\mu \psi_{\vk i}}+\braket{\partial_\mu \psi_{\vk i}|\partial_\nu \psi_{\vk i}}
-\sum_{ij\in\cb}\braket{\partial_\nu\psi_{\vk i} | \psi_{\vk j}}\braket{\psi_{\vk j}|\partial_\mu \psi_{\vk i}}
-\braket{\partial_\mu \psi_{\vk i}|\psi_{\vk j}}\braket{\psi_{\vk j}|\partial_\nu\psi_{\vk i}}\\
  &&=\sum_{i\in\cb}
     \braket{\partial_\nu\psi_{\vk i}|\left(1-\sum_{j\in\cb} \ket{\psi_{\vk j}}\bra{\psi_{\vk j}}\right)|\partial_\mu \psi_{\vk i}}
   +\braket{\partial_\mu \psi_{\vk i}|\left(1-\sum_{j\in\cb}\ket{\psi_{\vk j}}\bra{\psi_{\vk j}}\right)|\partial_\nu \psi_{\vk i}}
\end{eqnarray}     
which concludes the proof that $g_{\mu\nu}$ with projectors is
equivalent to original definition.

Next we discretize the formula ~\ref{Eq:26}:
\begin{eqnarray}
&&g_{\mu\nu}(\vk)  =  \frac{1}{\Delta_\nu\Delta_\mu}\sum_{i j \in\cb}\Tr\left(
  \left(\ket{u_{\vk+\Delta_\mu, i}}\bra{u_{\vk+\Delta_\mu, i}}-\ket{u_{\vk i}}\bra{u_{\vk i}}\right)
  \left(\ket{u_{\vk+\Delta_\nu,j}}\bra{u_{\vk+\Delta_\nu,j}} -\ket{u_{\vk j}}\bra{u_{\vk j}}\right)
  \right)\\
  &&=  \frac{1}{\Delta_\nu\Delta_\mu}\sum_{i j \in\cb}
     \delta_{ij}+
    |\braket{u_{\vk+\Delta_\mu, i}|u_{\vk+\Delta_\nu,j}}|^2
   -|\braket{u_{\vk+\Delta_\nu,j}|u_{\vk i}}|^2 
   -|\braket{u_{\vk+\Delta_\mu, i}|u_{\vk j}}|^2
\end{eqnarray}
This last equation is clearly gauge invariant, because it only
contains absolute values squared. The diagonal part is
equal to the above derived Eq.~\ref{Eq:20}
\begin{eqnarray}
  g_{\mu\mu}(\vk)  =\frac{2}{\Delta_\mu\Delta_\mu}\sum_{i j \in\cb}
\delta_{ij}-|\braket{u_{\vk+\Delta_\mu,j}|u_{\vk i}}|^2 .
\end{eqnarray}
The advantage of this formula is of course that off-diagonal
components can be safely computed numerically.

It is convenient to project $g_{\mu\mu}$ to a band, and we define
$g^i_{\mu\mu}(\vk)$ such that
$g_{\mu\mu}=\sum_{i\in\cb}g_{\mu\mu}^i$. We also symmetrize with
respect to the positive and the negative directions, to obtain
%\begin{eqnarray}
%  g^j_{\mu\mu}(\vk)  =\frac{2}{(\Delta_\mu)^2}(1-\sum_{i \in\cb}|\braket{u_{\vk+\Delta_\mu,j}|u_{\vk i}}|^2) .
%\end{eqnarray}
%or if we symmetrize 
\begin{eqnarray}
  g^j_{\mu\mu}(\vk)  =\frac{1}{(\Delta_\mu)^2}
  \left(1-\sum_{i\in\cb}|\braket{u_{\vk+\Delta_\mu,j}|u_{\vk i}}|^2+
          1-\sum_{i\in\cb}|\braket{u_{\vk-\Delta_\mu,j}|u_{\vk i}}|^2
  \right) .
\end{eqnarray}


\subsection{Regularization}

It turns out that $g_{\mu\nu}(\vk)$ is often diverging when two bands
almost touch. However, such divergence does not contribute to the
superfluid weight. The equation for superfluid weight contains factors
that cancel such divergence, which however vanish in the limit of
large gap. For general case with crossing bands it is more
appropriately to not take the limit of large gap, hence we will go
back to Eq.20 in PRB 95, 024515 (2017). It shows that
\begin{eqnarray}
  D^{\mu\nu}_{geom}  &&=\sum_{\vk,m\ne n}
  \Delta^2\left[\frac{\tanh(\beta/2
                        \sqrt{\varepsilon_m^2+\Delta^2})}{\sqrt{\varepsilon_m^2+\Delta^2}}-\frac{\tanh(\beta/2
                        \sqrt{\varepsilon_n^2+\Delta^2})}{\sqrt{\varepsilon_n^2+\Delta^2}}
  \right]
  \nonumber\\
  &&\times\frac{\varepsilon_n-\varepsilon_m}{\varepsilon_n+\varepsilon_m}
%
     \left[\braket{\partial_\nu\psi_{\vk m}|\psi_{\vk n}}\braket{\psi_{\vk n}|\partial_\mu \psi_{\vk m}}
   +\braket{\partial_\mu \psi_{\vk m}|\psi_{\vk n}}\braket{\psi_{\vk n}|\partial_\nu \psi_{\vk m}}\right]
 % 
\end{eqnarray}
Here energies are measured from the Fermi level.

We will take $T\rightarrow 0$ and compute:
%and adopt the second line in this expression as the modified
%metric tensor, i.e.,
\begin{eqnarray}
  D^{\mu\nu}_{geom} =\Delta^2
  \sum_{m\in\cb,n\notin\cb}
  \left( \frac{1}{\sqrt{\varepsilon_m^2+\Delta^2}}-
  \frac{1}{\sqrt{\varepsilon_n^2+\Delta^2}}
  \right)
  \frac{\varepsilon_n-\varepsilon_m}{\varepsilon_n+\varepsilon_m}
    \left[ \braket{\partial_\nu\psi_{\vk m}|\psi_{\vk n}}\braket{\psi_{\vk n}|\partial_\mu \psi_{\vk m}}
   +\braket{\partial_\mu \psi_{\vk m}|\psi_{\vk n}}\braket{\psi_{\vk n}|\partial_\nu \psi_{\vk m}}\right]
\end{eqnarray}  
Clearly, in the limit of large gap between the bands in $\cb$ and
those outside $\cb$, it must be that the ratio of energies approaches
$\frac{\varepsilon_n-\varepsilon_m}{\varepsilon_n+\varepsilon_m}\rightarrow
1$, because $\varepsilon_m$ remains small (in the gap), and $\varepsilon_n$
is of the order of the gap.


Next we prove that the gauge invariant formula has similar form to the
one used before, namely,
\begin{eqnarray}
  D_{geom,\mu\nu}=  \sum_{m\in\cb,n\notin\cb}
 \left( \frac{\Delta^2}{\sqrt{\varepsilon_m^2+\Delta^2}}-
  \frac{\Delta^2}{\sqrt{\varepsilon_n^2+\Delta^2}}
  \right)
  \frac{\varepsilon_m-\varepsilon_n}{\varepsilon_m+\varepsilon_n}
  \Tr\left[ \partial_\mu(\ket{\psi_{\vk n}}\bra{\psi_{\vk n}})  \partial_\nu (\ket{\psi_{\vk m}}\bra{\psi_{\vk m}})
  \right]
\end{eqnarray}  
The proof goes in the same way as in Eq.~\ref{Eq:projectorProof}, the
only difference being that sums over $n,m$ are different than before,
and therefore $\delta_{n,m}$ here vanishes.
But let us repeat the proof. Here we will omit $\frac{1}{\sqrt{...}}$,
and get them back at the end.
\begin{eqnarray}
  \widetilde{g}_{\mu\nu}&=&  \sum_{m\in\cb,n\notin\cb}
  \frac{\varepsilon_m-\varepsilon_n}{\varepsilon_m+\varepsilon_n}
  \Tr\left[ \partial_\mu(\ket{\psi_{\vk n}}\bra{\psi_{\vk n}})  \partial_\nu (\ket{\psi_{\vk m}}\bra{\psi_{\vk m}})
  \right]\\
&=&  \sum_{m\in\cb,n\notin\cb}
  \frac{\varepsilon_m-\varepsilon_n}{\varepsilon_m+\varepsilon_n}
  \Tr\left[
  (\ket{\partial_\mu\psi_{\vk n}}\bra{\psi_{\vk n}}+\ket{\psi_{\vk n}}\bra{\partial_\mu\psi_{\vk n}})
  (\ket{\partial_\nu \psi_{\vk m}}\bra{\psi_{\vk m}}+\ket{\psi_{\vk m}}\bra{\partial_\nu \psi_{\vk m}})
  \right]
  \nonumber\\
&=&  \sum_{m\in\cb,n\notin\cb}
  \frac{\varepsilon_m-\varepsilon_n}{\varepsilon_m+\varepsilon_n}
  \left[
   \braket{\psi_{\vk n}|\partial_\nu \psi_{\vk m}}\braket{\psi_{\vk m}|\partial_\mu\psi_{\vk n}}+
    \braket{\partial_\mu\psi_{\vk n}|\partial_\nu \psi_{\vk  m}}\delta_{n m}+
   \delta_{n m}\braket{\partial_\nu \psi_{\vk m}|\partial_\mu\psi_{\vk n}}+
   \braket{\partial_\mu\psi_{\vk n}|\psi_{\vk m}}\braket{\partial_\nu \psi_{\vk m}|\psi_{\vk n}}
  \right]
  \nonumber
\end{eqnarray}  
Because $m\in \cb$ and $n\notin\cb$ we can drop the middle two terms
containing $\delta_{nm}$. We also use $\braket{\psi_{\vk n}|\partial_\nu \psi_{\vk m}}=-\braket{\partial_\nu \psi_{\vk n}|\psi_{\vk m}}$
because $\partial_\nu \braket{\psi_{\vk n}|\psi_{\vk m}}=0$.
We thus have
\begin{eqnarray}
  \widetilde{g}_{\mu\nu}&=&  \sum_{m\in\cb,n\notin\cb}
  \frac{\varepsilon_m-\varepsilon_n}{\varepsilon_m+\varepsilon_n}
  \left[
   \braket{\psi_{\vk n}|\partial_\nu \psi_{\vk m}}\braket{\psi_{\vk m}|\partial_\mu\psi_{\vk n}}+
   \braket{\partial_\mu\psi_{\vk n}|\psi_{\vk m}}\braket{\partial_\nu \psi_{\vk m}|\psi_{\vk n}}
  \right]
\nonumber\\
&=&  -\sum_{m\in\cb,n\notin\cb}
  \frac{\varepsilon_m-\varepsilon_n}{\varepsilon_m+\varepsilon_n}
  \left[
   \braket{\psi_{\vk n}|\partial_\nu \psi_{\vk m}}\braket{\partial_\mu\psi_{\vk m}|\psi_{\vk n}}+
   \braket{\psi_{\vk n}|\partial_\mu\psi_{\vk m}}\braket{\partial_\nu \psi_{\vk m}|\psi_{\vk n}}
  \right]  
\nonumber\\
&=&  \sum_{m\in\cb,n\notin\cb}
  \frac{\varepsilon_n-\varepsilon_m}{\varepsilon_m+\varepsilon_n}
  \left[
   \braket{\partial_\mu\psi_{\vk m}|\psi_{\vk n}}\braket{\psi_{\vk n}|\partial_\nu \psi_{\vk m}}+
   \braket{\partial_\nu \psi_{\vk m}|\psi_{\vk n}}\braket{\psi_{\vk n}|\partial_\mu\psi_{\vk m}}
  \right]  
\end{eqnarray}  
which concludes the proof.

The finite difference formula for $\widetilde{g}_{\mu\nu}$ is
\begin{eqnarray}
D_{geom,\mu\nu}&=&
                   \frac{1}{\Delta_\mu\Delta_\nu}\sum_{m\in\cb,n\notin\cb}
\left( \frac{\Delta^2}{\sqrt{\varepsilon_m^2+\Delta^2}}-
  \frac{\Delta^2}{\sqrt{\varepsilon_n^2+\Delta^2}}
  \right)
  \frac{\varepsilon_m-\varepsilon_n}{\varepsilon_m+\varepsilon_n}
  \\                 
  &&\qquad \Tr\left[
  (\ket{u_{n,\vk+\Delta_\mu}}\bra{u_{n,\vk+\Delta_\mu}}-\ket{u_{n,\vk}}\bra{u_{n,\vk}})
  (\ket{u_{m,\vk+\Delta_\nu}}\bra{u_{m,\vk+\Delta_\nu}}-\ket{u_{m,\vk}}\bra{u_{m,\vk}})
  \right] \nonumber\\
&=&\frac{1}{\Delta_\mu\Delta_\nu}
    \sum_{m\in\cb,n\notin\cb}
\left( \frac{\Delta^2}{\sqrt{\varepsilon_m^2+\Delta^2}}-
  \frac{\Delta^2}{\sqrt{\varepsilon_n^2+\Delta^2}}
  \right)
    \frac{\varepsilon_n-\varepsilon_m}{\varepsilon_n+\varepsilon_m}\\
  &&
  \qquad\times\left[
    |\braket{u_{m,\vk+\Delta_\nu}|u_{n,\vk}}|^2
    +|\braket{u_{n,\vk+\Delta_\mu}|u_{m,\vk}}|^2
    -|\braket{u_{n,\vk+\Delta_\mu}|u_{m,\vk+\Delta_\nu}}|^2
    -|\braket{u_{n,\vk}|u_{m,\vk}}|^2
  \right]
\nonumber
\end{eqnarray}  
The diagonal expression is than
\begin{eqnarray}
  D_{geom,\mu\mu}=
\frac{1}{\Delta_\mu\Delta_\mu}
  \sum_{m\in\cb,n\notin\cb}
\left( \frac{\Delta^2}{\sqrt{\varepsilon_m^2+\Delta^2}}-
  \frac{\Delta^2}{\sqrt{\varepsilon_n^2+\Delta^2}}
  \right)
  \frac{\varepsilon_n-\varepsilon_m}{\varepsilon_n+\varepsilon_m}
  \left[
    |\braket{u_{m,\vk+\Delta_\mu}|u_{n,\vk}}|^2
    +|\braket{u_{n,\vk+\Delta_\mu}|u_{m,\vk}}|^2
  \right]
\end{eqnarray}  


We might have problems to converge this numerically, because bands $n$
very far from the Fermi energy might contribute. To avoid that, we
should try to use the following formula
\begin{eqnarray}
  \widetilde{g}_{\mu\nu}= g_{\mu\nu} + (\widetilde{g}_{\mu\nu}-g_{\mu\nu})
\end{eqnarray}
where $(\widetilde{g}_{\mu\nu}-g_{\mu\nu})$ should have small
contribution from bands beyond some cutoff energy. We can compute the
difference in the following way:
\begin{eqnarray}
  \widetilde{g}_{\mu\nu}-g_{\mu\nu}=  \sum_{m\in\cb,n\notin\cb}
  \left(\frac{\varepsilon_m-\varepsilon_n}{\varepsilon_m+\varepsilon_n}-1\right)
  \Tr\left[ \partial_\mu(\ket{\psi_{\vk n}}\bra{\psi_{\vk n}})  \partial_\nu (\ket{\psi_{\vk m}}\bra{\psi_{\vk m}})
  \right]
\end{eqnarray}  
which can be implemented by
\begin{eqnarray}
  \widetilde{g}_{\mu\nu}-g_{\mu\nu}&=&
  \frac{1}{\Delta_\mu\Delta_\nu}
\sum_{m\in\cb,n\notin\cb}\left(\frac{\varepsilon_n-\varepsilon_m}{\varepsilon_n+\varepsilon_m}-1\right)
  \left[
    |\braket{u_{m,\vk+\Delta_\nu}|u_{n,\vk}}|^2
    +|\braket{u_{n,\vk+\Delta_\mu}|u_{m,\vk}}|^2
    -|\braket{u_{n,\vk+\Delta_\mu}|u_{m,\vk+\Delta_\nu}}|^2 
    -|\braket{u_{n,\vk}|u_{m,\vk}}|^2
  \right]
\nonumber
\end{eqnarray}  
where the last term is actually always vanishing.


%Here we used the formula
%\begin{eqnarray}
%\widetilde{g}_{\mu\nu}=  \sum_{m\in\cb,n\notin\cb}\frac{\varepsilon_m-\varepsilon_n}{\varepsilon_m+\varepsilon_n}
%  \Tr\left[ \partial_\mu(\ket{\psi_{\vk n}}\bra{\psi_{\vk n}})  \partial_\nu (\ket{\psi_{\vk m}}\bra{\psi_{\vk m}})
%  \right]
%\end{eqnarray}  
To prove this formula, we will use slightly different derivation. Before we proved the formula
\begin{eqnarray}
g_{\mu\nu}=  \sum_{m\in\cb,n\in\cb}
  \Tr\left[ \partial_\mu(\ket{\psi_{\vk n}}\bra{\psi_{\vk n}})  \partial_\nu (\ket{\psi_{\vk m}}\bra{\psi_{\vk m}})
  \right]
\end{eqnarray}  
which can also be cast in the following way
\begin{eqnarray}
g_{\mu\nu}=  -\sum_{m\in\cb,n\notin\cb}
  \Tr\left[ \partial_\mu(\ket{\psi_{\vk n}}\bra{\psi_{\vk n}})  \partial_\nu (\ket{\psi_{\vk m}}\bra{\psi_{\vk m}})
  \right].
\end{eqnarray}  
This is because 
\begin{eqnarray}
0=  \sum_{m\in\cb}
  \Tr\left[ \partial_\mu\left(\sum_n\ket{\psi_{\vk n}}\bra{\psi_{\vk n}}\right)  \partial_\nu (\ket{\psi_{\vk m}}\bra{\psi_{\vk m}})
  \right].
\end{eqnarray}  
We can than clearly establish that 
\begin{eqnarray}
\widetilde{g}_{\mu\nu}-g_{\mu\nu}=  \sum_{m\in\cb,n\notin\cb}\left(1-\frac{\varepsilon_n-\varepsilon_m}{\varepsilon_n+\varepsilon_m}\right)
  \Tr\left[ \partial_\mu(\ket{\psi_{\vk n}}\bra{\psi_{\vk n}})  \partial_\nu (\ket{\psi_{\vk m}}\bra{\psi_{\vk m}})
  \right]
\end{eqnarray}  
which has the finite difference formula derived before. This concludes
the proof.

Finally, let us repeat all formulas that we will implement:
\begin{eqnarray}
  g^m_{\mu\nu} &=&   \frac{1}{\Delta_\mu\Delta_\nu}
\sum_{n\in\cb}
  \left[
    |\braket{u_{n,\vk}|u_{m,\vk}}|^2 
    +|\braket{u_{n,\vk+\Delta_\mu}|u_{m,\vk+\Delta_\nu}}|^2 
    -|\braket{u_{m,\vk+\Delta_\nu}|u_{n,\vk}}|^2
    -|\braket{u_{n,\vk+\Delta_\mu}|u_{m,\vk}}|^2
  \right]
\nonumber\\
  \widetilde{g}^m_{\mu\nu} &=&   \frac{1}{\Delta_\mu\Delta_\nu}
\sum_{n\notin\cb}\left(\frac{\varepsilon_n-\varepsilon_m}{\varepsilon_n+\varepsilon_m}\right)
  \left[
    |\braket{u_{m,\vk+\Delta_\nu}|u_{n,\vk}}|^2
    +|\braket{u_{n,\vk+\Delta_\mu}|u_{m,\vk}}|^2
    -|\braket{u_{n,\vk+\Delta_\mu}|u_{m,\vk+\Delta_\nu}}|^2 
    -|\braket{u_{n,\vk}|u_{m,\vk}}|^2
  \right]
\nonumber\\
  \widetilde{g}^m_{\mu\nu}-g^m_{\mu\nu}&=&
  \frac{1}{\Delta_\mu\Delta_\nu}
\sum_{n\notin\cb}\left(\frac{\varepsilon_n-\varepsilon_m}{\varepsilon_n+\varepsilon_m}-1\right)
  \left[
    |\braket{u_{m,\vk+\Delta_\nu}|u_{n,\vk}}|^2
    +|\braket{u_{n,\vk+\Delta_\mu}|u_{m,\vk}}|^2
    -|\braket{u_{n,\vk+\Delta_\mu}|u_{m,\vk+\Delta_\nu}}|^2 
    -|\braket{u_{n,\vk}|u_{m,\vk}}|^2
  \right]
\nonumber
\end{eqnarray}  

\newpage
\subsection{Integral}

The integral over the BZ can  be defined in the following way:
\begin{eqnarray}
  g_{\mu\nu}^\cb=\frac{1}{2\pi}\int_{BZ} d^3k g_{\mu\nu}(\vk)  =
  \frac{(2\pi)^2}{V_{cell}} \int_{BZ}\frac{V_{cell} d^3k}{(2\pi)^3} g_{\mu\nu}(\vk) 
=\frac{(2\pi)^2}{V_{cell}} \frac{1}{N_k} \sum_{\vk}g_{\mu\nu}(\vk) 
\end{eqnarray}  

If the integral is carried over in cartesian coordinates, and
$\Delta_\mu$ and $\Delta_\nu$ are displacements in cartesian
coordinates, the expressions above can be straightforwardly
implemented.

However, if we want to use the existing software, which implements the
input to Wannier90, we are forced to use derivative in the lattice
coordinates and uniform mesh in momentum space. Namely, Wannier90 uses
the following matrix elements
\begin{eqnarray}
M_{mn}^{\vk,\vb}=  \braket{u_{m\vk}|u_{n,\vk+\vb}}
\end{eqnarray}  
where $\vb$ is a small vector connecting nearest neighbors, and $\vk$
is vector in the first Brillouin zone.

If the integral is carried out in non-orthogonal lattice systems, we need to take into account matric of the lattice coordinates. We have
\begin{eqnarray}
  \eta_{ij}=\vec{b}_i \cdot \vec{b}_j = (BR2^T\cdot BR2)_{ij}\\
  \eta^{ij} = \left(\eta^{-1}\right)_{ij}
\end{eqnarray}
The gradient is computed as
\begin{eqnarray}
\sum_i \frac{\partial f}{\partial k_i} \vec{e_i}= \sum_{ij}\frac{\partial f}{\partial b_i}\eta^{ij} \vec{b}_j  
\end{eqnarray}
hence
\begin{eqnarray}
 M_{\mu\nu}^{\cb}(\vk)  =2\sum_{i\in \cb}
   \braket{\frac{\partial}{\partial k_{p}}\psi_{\vk i}|(1-\sum_{j\in \cb}\ket{\psi_{\vk j}}\bra{\psi_{\vk j}})|\frac{\partial}{\partial k_{q}}\psi_{\vk i}}\eta^{p r}\eta^{q t} b^{r}_{\mu} b^t_{\nu}
\end{eqnarray}
The trace is easier
\begin{eqnarray}
 \sum_{\mu}M_{\mu\mu}^{\cb}(\vk)  =2\sum_{i\in \cb}
   \braket{\frac{\partial}{\partial k_{p}}\psi_{\vk i}|(1-\sum_{j\in \cb}\ket{\psi_{\vk j}}\bra{\psi_{\vk j}})|\frac{\partial}{\partial k_{q}}\psi_{\vk i}}\eta^{p r}\eta^{q t} \eta_{t r}\\
=2\sum_{i\in \cb}
   \braket{\frac{\partial}{\partial k_{p}}\psi_{\vk i}|(1-\sum_{j\in \cb}\ket{\psi_{\vk j}}\bra{\psi_{\vk j}})|\frac{\partial}{\partial k_{q}}\psi_{\vk i}}\eta^{p q}
\end{eqnarray}
In this way we could reuse Wannier90 projectors $M_{mn}^{\vk,\vb}$.
For example, to compute 
\begin{eqnarray}
\frac{1}{2\pi}\int_{BZ} \sum_{p} g^{\cb}_{pp}(\vk) &=&   
\frac{(2\pi)^2}{ V_{cell}N_\vk}\sum_{\vk,m\in\cb,n\in\cb,\nu\mu}\frac{1}{\Delta_\mu\Delta_\nu}
  \left[
    \delta_{nm} 
    +|\braket{u_{n,\vk+\Delta_\mu}|u_{m,\vk+\Delta_\nu}}|^2 
    -|\braket{u_{n,\vk} |u_{m,\vk+\Delta_\nu}}|^2
    -|\braket{u_{m,\vk} |u_{n,\vk+\Delta_\mu}}|^2
  \right]\eta^{\mu\nu}
\nonumber\\
&=&\frac{(2\pi)^2}{ V_{cell}N_\vk}\sum_{\vk,m\in\cb,n\in\cb,\nu,\mu}\frac{1}{\Delta_\mu\Delta_\nu}
  \left[
    \delta_{nm} 
    +|M_{nm}^{\vk+\Delta_\mu,\Delta_\nu-\Delta_{\mu}}|^2 
    -|M_{nm}^{\vk,\Delta_\nu}|^2
    -|M_{mn}^{\vk,\Delta_\mu}|^2
  \right]\eta^{\mu\nu}
\end{eqnarray}

\subsubsection{Relation to Marzari-Vanderbilt}

The integral of the quantum geometric tensor is equal to the
invariant part of the Marzari-Vanderbilt spread functional $\Omega_{I}$
\begin{equation}
\Omega_I = \frac{V}{(2\pi)^3}\int_{BZ}d^3k\sum_\mu g_{\mu\mu}(\vk)
\end{equation}
where
\begin{equation}
\Omega_I = \sum_n \braket{r^2}_n - \sum_{\vR,m}|\braket{\vR m|\vr|0 n}|^2
\end{equation}  
They also point out that the integral along the path of GQT is a cumulative change of character of a band.


\subsubsection{Relation to Coulomb interaction}
The Coulomb interaction at small $\vq$ is also related to geometric tensor. The Coulomb interaction between bands can be written as
\begin{eqnarray}
V_\vq(\vk i j \vk' i'j') \equiv \braket{\psi_{\vk i}\psi^*_{\vk+\vq j} |\frac{e^{i\vq \vr}}{\sqrt{V}}}  \frac{4\pi}{\vq^2}
\braket{\frac{e^{i\vq \vr'}}{\sqrt{V}}|\psi_{\vk' i'}\psi^*_{\vk'+\vq j'} } =\braket{u_{\vk i}|u_{\vk+\vq j}}  \frac{4\pi}{\vq^2}
\braket{u_{\vk'+\vq j'} |u_{\vk' i'}} 
\end{eqnarray}
The last Eq. is valid as long as $\vq$ is in the first BZ.
The diagonal part is
\begin{eqnarray}
 \sum_{ij\in \cb} \frac{4\pi}{\vq^2}\delta_{ij}-V_\vq(\vk i j \vk i j) = \sum_{ij\in \cb} \frac{4\pi}{\vq^2}(\delta_{ij}-
|\braket{u_{\vk+\vq j} |u_{\vk i}}|^2)=2\pi g^\cb_{\vq\vq}(\vk)
\end{eqnarray}  
and is valid only for very small $\vq$.

It seems that at small $\vq$, we have
\begin{eqnarray}
V_\vq(\vk i i \vk i i)   = \frac{4\pi}{\vq^2}-2\pi g^i_{\vq\vq}(\vk)
\end{eqnarray}
hence the periodic potential modifies the long range Coulomb
interaction by exactly the quantum metric. If the lattice is absent,
of course we just have the first term. But introduction of the lattice
screens the long-range Coulomb interaction of the band $i$ for its
quantum metric. Large quantum metric thus reduces log range Coulomb interaction.




Conventional contribution to D:
\begin{eqnarray}
  D_{conv,\mu\nu}=\sum_{m}\frac{\Delta^2}{({\varepsilon_m(\vk)^2+\Delta^2})^{3/2}}
  \partial_\nu \varepsilon_m(\vk) \partial_\mu \varepsilon_m(\vk)
\end{eqnarray}

\end{document}

Motivation. Most common used method to evaluate screened Coulomb interaction is constrained RPA. This just calculates the RPA screened interaction, and cuts out the bands that are treated by the model. If the fully screened Coulomb interaction is
$$W_{\vq} = (V_\vq^{-1}- P_{\vq})^{-1}$$
than cRPA interaction is
$$U_{\vq} = (V_\vq^{-1}-(P_{\vq}-P_{\vq}^\cb))^{-1}$$
Eventually we need to transform this quantity from product basis to Wannier basis, i.e., it is evaluated in the Wannier basis.







\begin{eqnarray}
M_{\mu\nu}^{\cb}(\vk) \approx 2\sum_{i\in \cb}
\braket{x_{\mu}\psi_{\vk i}|(1-\sum_{j\in \cb}\ket{\psi_{\vk j}}\bra{\psi_{\vk j}})|x_\nu \psi_{\vk i}}\\
\end{eqnarray}

\begin{eqnarray}
u_{\vk i}(\vr) \rightarrow e^{i\beta_i(\vk)}\widetilde{u}_{\vk i}(\vr)  
\end{eqnarray}

\begin{eqnarray}
\psi_{\vk j} = e^{i\vk \vr} u_{\vk j}  
\end{eqnarray}
